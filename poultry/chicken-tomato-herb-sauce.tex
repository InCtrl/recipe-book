\section{Chicken with Tomato Herb Pan Sauce}\index{poultry!chicken!tomato herb pan sauce}

\begin{center}
\noindent Total Time: 30 minutes

\noindent Yield: 4 servings

\vspace{1em}

Source: http://www.annies-eats.com/2011/09/08/chicken-with-tomato-herb-pan-sauce/
\end{center}

\subsection{Ingredients}
For the chicken:
\begin{itemize}
    \item 2-3 boneless, skinless chicken breasts, butterflied and halved (4-6 halves total)
    \item Salt and pepper
    \item 3/4 cup flour
\end{itemize}

\noindent For the sauce:
\begin{multicols}{2}
\begin{itemize}
    \item 2 tbsp. unsalted butter, softened
    \item 1 clove garlic, minced
    \item 1 1/2 tsp. fresh oregano, minced
    \item 1/2 tsp. sweet paprika
    \item Salt and pepper
    \item 2 tsp. olive oil
    \item 2 cups cherry or grape tomatoes (about 12 oz.)
    \item 1/3 cup dry white wine or chicken broth
    \item 1 tbsp. minced fresh parsley
\end{itemize}
\end{multicols}

\subsection{Preparation}
\begin{enumerate}
    \item Season both sides of the chicken breasts with salt and pepper.  Lightly dredge both sides of the chicken in the flour, shaking off the excess.  Set aside.
    \item In a small bowl, combine the butter, garlic, oregano, and paprika.  Season with salt and pepper to taste.  In a large skillet over medium-high heat, melt 1 tablespoon of the oregano butter with the olive oil.  Place the chicken breast halves in the skillet and cook until golden brown on each side and cooked through, about 3-4 minutes per side.  Transfer to a plate, cover loosely with foil, and set aside.
    \item Increase the heat to high and add the tomatoes to the skillet.  Cook, stirring occasionally, until the tomatoes begin to char and burst, about 5 minutes.  Add the remaining butter mixture to the pan.  Crush the tomatoes slightly to release their juices and continue stirring until the butter is melted.  Add the wine or broth to the pan, scraping the bottom to loosen the browned bits.  Cook for a minute more until well blended.
    \item Slice the chicken, transfer to serving plates, and top with the pan sauce.  Sprinkle with parsley and serve.
\end{enumerate}

