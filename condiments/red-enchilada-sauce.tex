\section{Red Enchilada Sauce}\index{condiments!red enchilada sauce}\label{red-enchilada-sauce}

\begin{center}
Prep Time: 5 min |
Cook Time: 15 min |
Total Time: 20 hour

\noindent Yield: 6-8 servings

\vspace{1em}

Source: http://www.gimmesomeoven.com/red-enchilada-sauce/
\end{center}

\subsection{Ingredients}
\begin{multicols}{2}
\begin{itemize}
    \item 2 Tbsp. vegetable or canola oil
    \item 2 Tbsp. all-purpose or gluten-free flour
    \item 4 Tbsp. chili powder (*not cayenne! also, see note below if you do not like heat*)
    \item 1/2 tsp. garlic powder
    \item 1/2 tsp. salt
    \item 1/4 tsp. cumin
    \item 1/4 tsp. oregano
    \item 2 cups chicken or vegetable stock
\end{itemize}
\end{multicols}

\subsection{Preparation}
\begin{enumerate}
    \item Heat oil in a small saucepan over medium-high heat. Add flour and stir together over the heat for one minute. Stir in the remaining seasonings (chili powder through oregano). Then gradually add in the stock, whisking constantly to remove lumps. Reduce heat and simmer 10-15 minutes until thick.
    \item Use immediately or refrigerate in an air-tight container for up to two weeks.
\end{enumerate}
* (Note from the original author) I consider this sauce pretty mild.
But if you are wary about heat/spice in your sauce, I would begin with 2 tablespoons chili powder and add more from there once the sauce has reached a simmer if you'd like.
Again, I am using chili powder for this recipe, not cayenne.
From the comments, it sounds as though chili powders vary from country to country.
But the traditional American chili powder is fairly mild, and should not be overly spicy.

