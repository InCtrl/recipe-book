\section{Fried Rice}\index{grains!rice!fried rice}

\begin{center}
Prep Time: 10 min |
Cook Time: 15 min |
Total Time: 25 min

\noindent Yield: 4-6 servings

\vspace{1em}

    Source: \url{https://www.gimmesomeoven.com/fried-rice-recipe/}
\end{center}

\subsection{Ingredients}
\begin{multicols}{2}
\begin{itemize}
  \item 3 tablespoons butter, divided
  \item 2 eggs, whisked
  \item 2 medium carrots, peeled and diced
  \item 1 small white onion, diced
  \item 1/2 cup frozen peas
  \item 3 cloves garlic, minced
  \item salt and black pepper
  \item 4 cups cooked and chilled rice (I prefer short-grain white rice)
  \item 3 green onions, thinly sliced
  \item 3-4 tablespoons soy sauce, or more to taste
  \item 2 teaspoons oyster sauce (optional)
  \item 1/2 teaspoons toasted sesame oil
\end{itemize}
\end{multicols}

\subsection{Preparation}
\begin{enumerate}
  \item Heat 1/2 tablespoon of butter in a large saut\'{e} pan* over medium-high heat until melted. Add egg, and cook until scrambled, stirring occasionally. Remove egg, and transfer to a separate plate.
  \item Add an additional 1 tablespoon butter to the pan and heat until melted. Add carrots, onion, peas and garlic, and season with a generous pinch of salt and pepper. Saut\'{e} for about 5 minutes or until the onion and carrots are soft. Increase heat to high, add in the remaining 1 1/2 tablespoons of butter, and stir until melted. Immediately add the rice, green onions, soy sauce and oyster sauce (if using), and stir until combined. Continue saut\'{e}ing for an additional 3 minutes to fry the rice, stirring occasionally.  (I like to let the rice rest for a bit between stirs so that it can crisp up on the bottom.)  Then add in the eggs and stir to combine. Remove from heat, and stir in the sesame oil until combined.  Taste and season with extra soy sauce, if needed.
  \item Serve immediately, or refrigerate in a sealed container for up to 3 days.
\end{enumerate}

\subsection{Notes}
*Saut\'{e} pan: If you happen to own a nonstick or cast-iron pan, I would recommend it for this recipe.  But that said, any pan that you have can work - you may just have to be a bit more vigilant with stirring so that the rice and eggs don't stick.

\subsection{Easy Fried Rice Variation}
The sky's the limit when it comes to homemade fried rice variations, so feel free to get creative and use up other leftover ingredients you may happen to have on hand.
That said, here are some classic add-ins:
\begin{itemize}
  \item Chicken Fried Rice: You can either saut\'{e} some chicken in a separate saut\'{e} pan while making your fried rice.  Then shred or dice and add to your fried rice.  Or for a shortcut, I like to shred a rotisserie chicken.  Or even better, shred leftovers from my favorite baked chicken breasts recipe.
  \item Pork Fried Rice: Saut\'{e} a boneless pork chop in a separate saut\'{e} pan while making your fried rice.  Then dice and add to your rice.
  \item Beef Fried Rice: Saut\'{e} steak or brown ground beef in a separate saut\'{e} pan while making your fried rice.  Then crumble or dice and add to your rice.
  \item Shrimp Fried Rice: Saut\'{e} half a pound of peeled, raw shrimp in separate saut\'{e} pan while making your fried rice.  Then add the shrimp to your rice.
  \item Vegetable Fried Rice: Any stir-fry friendly veggies would be great in fried rice!  Just saut\'{e} at the same time that you cook the onions, carrots, peas and garlic.  Then stir to combine with the fried rice.
  \item Kimchi Fried Rice: Fresh kimchi adds a major flavor boost to fried rice.  Just chop and stir it in to make kimchi fried rice.
  \item Pineapple Fried Rice: Fresh pineapple can be traditional in Chinese or Thai fried rice.  Just chop and stir it in to combine.
\end{itemize}
Also, in lieu of using traditional white rice in this recipe, feel free to make:
\begin{itemize}
  \item Fried Brown Rice: For a healthier twist, feel free to use cooked brown rice in this recipe instead of white rice.
  \item Quinoa Fried Rice: Here is my favorite recipe.
\end{itemize}
