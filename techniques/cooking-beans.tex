\section{How to Cook Beans}\index{techniques!how to cook beans}
Source: http://www.thekitchn.com/how-to-cook-beans-43766

\subsection{Introduction}
Beans are an overlooked, sadly underused kitchen staple.
 The first time we tried Rancho Gordo beans we were stunned then seduced by the rich, smoky depths of flavor in a simple pot of beans.

Our recipe for Basil Parmesan Pot Beans has been popular ever since we posted it, but we've had some questions on how to cook beans.
Here is how to cook dried beans and how to cook fresh beans.

Beans are really quite simple: you cook them in water until they're soft as you like them.
Really!
 That's it.

OK, there are a few more tips we've found, and some general time guidelines.

\begin{enumerate}
    \item Use as little water as possible.
          We don't want our beans to get soggy or ultra-soft, and we really don't want to dilute their rich pot liquor, the incredibly flavorful liquid that comes off as they cook.
          So we cook beans very slowly over low heat, only adding water if they start to dry up.

          So when we are cooking pot beans we only add enough water to just cover the beans.
          Bring it to a simmer then reduce the heat and cook, uncovered, as low as you can.

    \item Don't mess with the beans.
          Sometimes we fry a little bacon or onion and garlic, then add the beans and water.
          But often even this is too much. Again, well-grown heirloom beans have incredible ranges of flavor - far, far from the homogenized black, kidney, and white beans in cans.
          There are nuances and subtleties that surpass meat.
          Try a couple pots of beans with just salt and pepper - you'll be surprised at what you taste.

    \item Don't forget the salt! Beans need a lot of salt.
          They have immense natural flavor, but they need some salt to bring it out, and they absorb quite a bit before it starts show through.
          We were not impressed by our first small pot of beans until we added salt to taste.
          Don't underestimate the salt you need.

        Add a teaspoon of salt to the cooking water and more to taste, if necessary, in the last half hour of cooking.
        Give them time since it takes a little while for the beans to absorb salt.
\end{enumerate}

\subsection{Soaking}
Rinsing then soaking overnight in clean water will reduce the cooking time for most beans, although good fresh dried beans are less in need of a soak.

\subsection{How long to cook dried beans}
This obviously depends on the bean, but usually you're looking at about 2-4 hours.
We plan on at least 3 hours for most of our favorite dried beans, like Good Mother Stallard from Rancho Gordo (better than many kinds of meat!).
Cover with water and simmer on an evening when you're doing other things.
Refrigerate and eat over the next several days.

\subsection{How long to cook fresh beans}
Freshly hulled beans, like the lovely cranberry beans above, will cook in about 45 minutes or less.
Again, it depends on the bean and how fresh it is.

We're trying to eat fresher and healthier this spring, and we're also working on eating less meat.
Beans are a great way to go - we haven't even begun to explore all the varieties we want to try!
