\section{Grilled Venison Tenderloin}\index{venison!tenderloin}

Source: recipetips.com - A tender, delicious and quick-cooking cut. Size may vary depending on the size of the animal.

\noindent Prep. Time: 1 hour 30 minutes |
Total Time: ~2 hours | 
Container: Grill

\subsection{Ingredients}
\begin{multicols}{2}
\begin{itemize}
    \item 1 pound venison tenderloin
    \item 1 cup red wine
    \item 2 tablespoons olive oil
    \item 1 tablespoon soy sauce
    \item 1 clove garlic, mashed
    \item 1 tablespoon fresh rosemary, chopped or 1 teaspoon dried and crumbled
    \item 1 tablespoon current jam or jelly (substitute grape)
\end{itemize}
\end{multicols}

\subsection{Preparation}
\begin{enumerate}
    \item Trim meat: remove all visible fat and most of the silver skin (translucent membrane). If loin has a long tapered end, curl the thin end back and toothpick it in place so it doesn't overcook.
    \item Mix together wine, olive oil, soy sauce, garlic and rosemary; put into zipper-top plastic bag. Add loin and marinate, refrigerated, 1 - 2 hours, turning bag occasionally (longer if meat is from an older animal).
    \item Start grill. Remove meat from marinade and pat dry. Bring to room temperature before grilling.
    \item While grill heats, pour marinade into a small pan. Over high heat, cook down to about 1/2 cup. Stir in current jam or jelly. Strain, discarding solids.
    \item Grill meat over direct heat, turning to brown all sides. Allow 6-8 minutes per inch of thickness. Use an instant-read thermometer (125° F is medium rare) or make a small cut to check doneness. Don't overcook or meat will be dry and tasteless.
    \item Let rest on warm platter a few minutes to distribute juices within the meat. Serve whole or sliced into 1'' rounds with the current sauce.

\end{enumerate}
